\section{The algorithm}
This section will go through the Efficient Graph-Based Image Segmentation
algorithm introduced by Pedro F. Felzenswalb and Daniel P. Huttenlocher \cite{felzenszwalb2004}.
To help the explanation throughout the chapter four images will be used, that is figure \ref{fig:smile-rgb},
\ref{fig:smile-red}, \ref{fig:smile-green} and \ref{fig:smile-blue}.

The algorithm is constructed in a general fashion which makes it possible to use
under many different settings. The key parts needed are an undirected graph and
some weight on each edge in the graph. As an example to this figure \ref{fig:smile-red}
can help, the red channel from the "smile" image. The image is 10x10 pixels in
size and construction of the undirected weighted graph is done by comparing
neighbouring pixels with each other. All pixels except the ones around the border
are connected to 8 neighbours (right,left,up,down and all four corners).
In other words the connectivity is chosen to be 8. Other types of connectevity
is 4-adjacency or mixed adjacency.

% TODO: Short example with image about connectivity?

When building the graph each edge get a weight that is the difference between
intensity in connected pixels. Pixel \(I(0,0)\) is mostly black so a good estimate
is that it has an intensity level of 10. The neighbour to the right \(I(0,1)\) has
the same intensity but the two neighbours downwards \(I(1,0)\) and \(I(1,1)\) are
white. An estimate for the white color is 250 in intensity. The edge weight is
now calculated to 0 between the two black pixels and 240 between the black and the
two white pixels.

\begin{figure}[ht]
    \begin{minipage}[t]{0.45\linewidth}
        \centering
        \includegraphics[width=\textwidth]{images/smile/smile-original.jpg}
        \caption{Smile, RGB color image}
        \label{fig:smile-rgb}
    \end{minipage}
    \hspace{0.5cm}
    \begin{minipage}[t]{0.45\linewidth}
        \centering
        \includegraphics[width=\textwidth]{images/smile/smile-red-channel.jpg}
        \caption{Smile, red channel.}
        \label{fig:smile-red}
    \end{minipage}
\end{figure}
\begin{figure}[ht]
    \begin{minipage}[t]{0.45\linewidth}
        \centering
        \includegraphics[width=\textwidth]{images/smile/smile-green-channel.jpg}
        \caption{Smile, green channel.}
        \label{fig:smile-green}
    \end{minipage}
    \hspace{0.5cm}
    \begin{minipage}[t]{0.45\linewidth}
        \centering
        \includegraphics[width=\textwidth]{images/smile/smile-blue-channel.jpg}
        \caption{Smile, blue channel.}
        \label{fig:smile-blue}
    \end{minipage}
\end{figure}

With the undirected weighted graph built it's time to take a closer look at the
segmentation. The goal with the algorithm is to be fast and {\em "Capture perceptually
important groupings or regions, which often reflect global aspects of the image"} \cite{felzenszwalb2004}.
One easy way to segment an image is to say that all edges that are less than some
constant \(k\) should belong to the same region. So in the example image, figure \ref{fig:smile-red},
it would work pretty well as all the regions that are visually considered to belong
together would be grouped together. The problem arises as soon as this is tried
on a "real" image that has not been created on a computer. Remember the discussion
about noise from earlier chapter. Noise in just one pixel, e.g. the center
pixel in one of the eyes could cause the algorithm (with constant \(k\)) to divide
the eye into 5 different regions.

Instead of a constant that decides whether a pixel should belong to a region or
not Felzenswalb and Huttenlocher have come up with a predicate that is variable
and takes the neighbouring regions into account. To continue explaining the
algorithm some terminology and a more mathematical approach is needed. The following
paragraphs until the end of this chapter are the key parts from the original paper \cite[ch. 3.1 Pairwise
Region Comparison Predicate]{felzenszwalb2004}.

The goal is to decide whether there should be a boundary between two pixels or
two pixels should belong to the same component. A component is a connected region of one
or more pixels. Let \(D\) be the predicate that is true if there should be a
boundary or false if the two pixels should belong to the same component. Before
defining \(D\) two other definitions are needed. First of all the definition
of {\em internal difference}.
\begin{equation}
    \label{eq:internalDifference}
    \text{Int}(C) = \underset{e \in \text{MST}(C,E)}{\text{max}} w(e)
\end{equation}

\(Int(C)\), the internal difference in a component is the maximum weight of
all edges within that component. As an example if three pixels, that belongs to the same component,
are connected with the edge weights 50, 100 and 150. The internal difference
would be 150. Next up is the {\em difference between} components.
\begin{equation}
    \label{eq:differenceBetween}
    \text{Dif}(C_1,C_2) = \underset{v_i \in C_1,v_j \in C_2, (v_i,v_j) \in E}{\text{min}} w((v_i,v_j))
\end{equation}
The difference between two components is the minimum weight edge between the
components. With the internal difference and the difference between two components,
it's time to define the predicate.
\begin{equation}
    \label{eq:predicate}
    D(C_1,C_2) =
    \begin{cases}
        true & \text{if } Dif(C_1,C_2) > MInt(C_1,C_2) \\
        false & otherwise
    \end{cases}
\end{equation}
The predicate says that there is reason for a border (true) if the difference
between two components is larger than the minimum internal difference \(MInt\)
in the two components. In equation \ref{eq:internalDifference} the internal
difference is defined for one component though with that definition the predicate
in equation \ref{eq:predicate} is not very good for small components. So instead
of just selecting the minimum internal difference in components a threshold function
is added.
\begin{equation}
    \label{eq:minimumInternal}
    \text{MInt}(C_1,C_2) = \text{min}(\text{Int}(C_1) + \tau(C_1),\text{Int}(C_2) + \tau(C_2))
\end{equation}
\begin{equation}
    \label{eq:threshold}
    \tau(C) = \frac{k}{|C|}
\end{equation}
The threshold function will depend on the size of the components being compared
and will be most important for small components and as the components grove the
weight of the threshold function will fade away. This finishes of the definition
of the predicate and it's time to look at the actual steps taken in the
algorithm.

The efficient graph-based image segmentation algorithm takes a graph as input
\(G = (V,E)\) with \(n\) vertices and \(m\) edges. Remember from earlier chapter
that in image segmentation it's not a new image that is the output instead it's
the regions that have been found in the image. The output in the following text
is defined as \(S = (C_1,...,C_r)\). The algorithm has five main steps and the
steps are as follows.
\begin{enumerate}
    \item Sort all edges \(E\) in non-decreasing order.
    \item All pixels are their own component from start.
    \item Repeat step 4 for all edges.
    \item Select the edge with the lowest weight and use the predicate defined in
        equation \ref{eq:predicate} to decide whether the components should belong
        to the same component or leave them in separate components. Merge components
        if needed.
    \item Return S with all components.
\end{enumerate}
For a more detailed explanation of the algorithm and its pros and cons the
original paper which introduced the algorithm is listed in the references as \cite{felzenszwalb2004}.

% TODO: Add basics about Kruskal's minimum spanning tree.

% TODO: Add detailed gaussion description in digital image processing and
% then describe the difference here what it means to increase sigma value.

% TODO: Show an simple color image that has some differences when it
% comes to 4 or 8 connectness.

% \subsection{Building the graph}
% \subsection{Threshold function}
% What about taking into account width and breadth but crossings wouldn't
% count and curved roads would not be found.
