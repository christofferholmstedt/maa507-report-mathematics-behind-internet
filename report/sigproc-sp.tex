% THIS IS SIGPROC-SP.TEX - VERSION 3.1
% WORKS WITH V3.2SP OF ACM_PROC_ARTICLE-SP.CLS
% APRIL 2009
%
% It is an example file showing how to use the 'acm_proc_article-sp.cls' V3.2SP
% LaTeX2e document class file for Conference Proceedings submissions.
% ----------------------------------------------------------------------------------------------------------------
% This .tex file (and associated .cls V3.2SP) *DOES NOT* produce:
%       1) The Permission Statement
%       2) The Conference (location) Info information
%       3) The Copyright Line with ACM data
%       4) Page numbering
% ---------------------------------------------------------------------------------------------------------------
% It is an example which *does* use the .bib file (from which the .bbl file
% is produced).
% REMEMBER HOWEVER: After having produced the .bbl file,
% and prior to final submission,
% you need to 'insert'  your .bbl file into your source .tex file so as to provide
% ONE 'self-contained' source file.
%
% Questions regarding SIGS should be sent to
% Adrienne Griscti ---> griscti@acm.org
%
% Questions/suggestions regarding the guidelines, .tex and .cls files, etc. to
% Gerald Murray ---> murray@hq.acm.org
%
% For tracking purposes - this is V3.1SP - APRIL 2009

\documentclass{acm_proc_article-sp}
\usepackage{url}
\begin{document}

\title{Minimum spanning tree and image segmentation\titlenote{This report was written in the spring of 2013 in the advanced level course MAA507 "Mathematics behind internet" at M\"{a}lardalen University, Sweden.}}
%\subtitle{[Extended Abstract]
%\titlenote{A full version of this paper is available as
%\textit{Author's Guide to Preparing ACM SIG Proceedings Using
%\LaTeX$2_\epsilon$\ and BibTeX} at
%\texttt{www.acm.org/eaddress.htm}}}
%
% You need the command \numberofauthors to handle the 'placement
% and alignment' of the authors beneath the title.
%
% For aesthetic reasons, we recommend 'three authors at a time'
% i.e. three 'name/affiliation blocks' be placed beneath the title.
%
% NOTE: You are NOT restricted in how many 'rows' of
% "name/affiliations" may appear. We just ask that you restrict
% the number of 'columns' to three.
%
% Because of the available 'opening page real-estate'
% we ask you to refrain from putting more than six authors
% (two rows with three columns) beneath the article title.
% More than six makes the first-page appear very cluttered indeed.
%
% Use the \alignauthor commands to handle the names
% and affiliations for an 'aesthetic maximum' of six authors.
% Add names, affiliations, addresses for
% the seventh etc. author(s) as the argument for the
% \additionalauthors command.
% These 'additional authors' will be output/set for you
% without further effort on your part as the last section in
% the body of your article BEFORE References or any Appendices.

\numberofauthors{1} 
% I've updated the number of authers ~ Christoffer 2012-10-14

%  in this sample file, there are a *total*
% of EIGHT authors. SIX appear on the 'first-page' (for formatting
% reasons) and the remaining two appear in the \additionalauthors section.
%
\author{
% You can go ahead and credit any number of authors here,
% e.g. one 'row of three' or two rows (consisting of one row of three
% and a second row of one, two or three).
%
% The command \alignauthor (no curly braces needed) should
% precede each author name, affiliation/snail-mail address and
% e-mail address. Additionally, tag each line of
% affiliation/address with \affaddr, and tag the
% e-mail address with \email.
%
% 1st. author
\alignauthor
Christoffer Holmstedt\\
       \email{christoffer.holmstedt@gmail.com}
}
% There's nothing stopping you putting the seventh, eighth, etc.
% author on the opening page (as the 'third row') but we ask,
% for aesthetic reasons that you place these 'additional authors'
% in the \additional authors block, viz.
% \additionalauthors{Additional authors: John Smith (The Th{\o}rv{\"a}ld Group,
%email: {\texttt{jsmith@affiliation.org}}) and Julius P.~Kumquat
%(The Kumquat Consortium, email: {\texttt{jpkumquat@consortium.net}}).}
%\date{30 July 1999}
% Just remember to make sure that the TOTAL number of authors
% is the number that will appear on the first page PLUS the
% number that will appear in the \additionalauthors section.

\maketitle
\begin{abstract}
This report goes through the Efficient Graph-Based Image Segmentation algorithm
and its usefulness on aerial images for road recognition with the purpose of
automating the process to create maps. Written as a part of an advanced level course
this report will give the reader an introduction to some basics of digital
image processing, the algorithm and how it all comes together in the application
of aerial images. The report concludes that without additional preprocessing or more
advanced definition of the threshold function the algorithm is of no use
in this application.
\end{abstract}

% A category with the (minimum) three required fields
\category{A.1}{General Literature}{Introductory and Survery}
%A category including the fourth, optional field follows...
% If we want to add another category (or several).
%\category{D.2.8}{CHANGE THIS Software Engineering}{Metrics}[complexity measures, performance measures]

\terms{TODO Theory}

\keywords{TODO Threat Models, Centralised systems, Decentralised systems, Censorship, Privacy, Natural disasters} % NOT required for Proceedings

\newpage
\section{Introduction}
OpenStreetMap is a map built by its users. Users go out to track roads and
other objects with GPS positioning and then adds that information together with
interesting metadata to the OpenStreetMap project. This information is then
available for all its users under a free and open license. The process of
gathering GPS positions for all objects in the world is quite time consuming
so instead of tracking objects by foot some aerial imagery sources are
available so users can stay at home and add objects to the project with the
aerial imagery as source.

Even to add objects manually from aerial images can sometimes be cumbersome and
tedious. This process can be automated and let the user approve with or
without small modications the end result. This would free up time for the
users to spend on other mapping activities such as adding house numbers,
locations of emergency services and other points of interests.

Another purpose to motivate this work is the problem of alignment for
different image sets. The entire world can not fit in a single image
and still include enough detail to see roads, houses and other objects.
Instead stitching is applied where e.g. fixed-wing airplanes fly at a fixed
altitude and in a constant speed and takes several images over a large area.
These images are then "stitched" together to create a large map with enough detail.

In theory this would create a perfect map over a large area though in practice
several problems arise. Some aerial images are photographed at an angle and
creates misalignment cause of this. Other image sets comes with erroneous metadata
such as wrong GPS positions. With GPS traces of roads already added to OpenStreetMap
and automatic road recognition from the aerial imagery it would be possible to
set individual offset for all images to match the positioning of GPS traces.

A third motivating factor is somewhat related to the first problem mentioned above.
When high detail images over areas which have not been mapped before (or old
images are updated with increased detail) are added to sources like Bing maps
it can take sometime before anyone notices the new image. If an automated
process could look through all new images and send notifications to active
users it could increase response time.

% Another usage would be to create a network of roads as node and map that to
% already traced GPS nodes/networks. This would give the possibility to
% automatically align misaligned aerial images for the OpenStreetMap project.

\begin{itemize}
% OpenStreetMap (OSM) is a crowdsourced map built by users collecting gps-traces
%and tracing objects through a few aerial imagery sources that are available.
%...something about JOSM...
%...about this report...
%...The structure of this report...
% In Appendix more thourough look at OpenStreetMap and Java OpenStreetMap Editor
% is given.
% In Appendix "How to use implementation"
% In Appendix "Algorithm recap, with suggested values for application"
%
% Write about 4/8/Mixed choices and how that can affect.
% Define Tau for my use case and explain in the general case.
%
% ...notes while reading medical phd report...
% Goal with this project is a fully automatic process, no human interaction.
    \item Background OSM why use bing maps? What are they used for? if no or only
        a few gps traces exists they can be used to improve the detail of the 
        open street map.
    \item Problems with using Bing maps?
    \item Images are not photgraphed straight above, this gives us different problems.
        \begin{itemize}
            \item Curvy roads
            \item misalignment
            \item
        \end{itemize}
    \item Idea, how to use bing maps, three different questions.
        \begin{itemize}
            \item Automatically calculate bing aerial imagery offset.
            \item
            \item
        \end{itemize}
    \item So alot of problems with Bing maps and image analysis exists
        we are only to look at the image segmentation problem
        and why speed matters in this application.
    \item
    \item
    \item
\end{itemize}



\section{Method}
Method text goes in here.


\section{Conclusions}
A big problem within the field of image segmentation is to define and judge
what a good segmentation is. If an algorithm identifies one part of an image
in the foreground while another identfies something else in the background, which
one is better is hard to say. What can be said is when the result is not adequate
for post processing such as road recognition in this case.

The image over Lule\.{a} clearly shows some roads while other parts of the
image are more vague even for the human eye. The segmentation gives some hints
to what could be roads but the detail is not enough for future work to be worth it.
The second image which basically is a highway with a junction through a forest
should be relatively easy to segment. Segmentation of the highway image shows
the road going through the central parts of the image and can be defined as
a good segmentation if it's only the highway that is to be identified. The smaller
roads are near impossible to see in the segmentation.

The two images tested in this case shows that the Efficient Graph-Based Image Segmentation
is not viable for this application. There may be future work that can be done
to create a better threshold function to detect thinner lines.

This project started out with the algorithm and application, the result as stated
above is not satisfactory so future work should start with more specific algorithms
for road recognition. Example of work that can be intresting to look closer to
is listed below.
\begin{itemize}
  \item {\em "Automatic Road Extraction from Aerial Images"} by John C. Trinder and
Yandong Wang \cite{trinder1998}.
  \item {\em "Detection of Roads in Aerial Images by
Using Edge Information"} by Wu-Ja Lin and Chih-Wei Tseng \cite{lin2012}.
  \item {\em "Urban Road Extraction from High-Resolution Optical Satellite Images"} by
Mohamed Naouai, Atef Hamouda and Christiane Weber \cite{naouai2010}.
  \item {\em "Unsupervised line network extraction in remote sensing using a polyline
      process"} by Caroline Lacoste, Xavier Descombes and Josiane Zerubia \cite{lacoste2009}.
\end{itemize}

% The eye sees some different shapes that is not there we make them up in our
% mind.
% \nocite{*}

%\end{document}  % This is where a 'short' article might terminate

% Just comment this out if we don't need it.
%\input{acknowledgment.tex}

%
% The following two commands are all you need in the
% initial runs of your .tex file to
% produce the bibliography for the citations in your paper.
\bibliographystyle{abbrv}
\bibliography{sigproc}  % sigproc.bib is the name of the Bibliography in this case
% You must have a proper ".bib" file
%  and remember to run:
% latex bibtex latex latex
% to resolve all references
%
% ACM needs 'a single self-contained file'!
%
%APPENDICES are optional
%\balancecolumns
%\appendix
%Appendix A
%\section{Headings in Appendices}
%The rules about hierarchical headings discussed above for
%the body of the article are different in the appendices.
%In the \textbf{appendix} environment, the command
%\textbf{section} is used to
%indicate the start of each Appendix, with alphabetic order
%designation (i.e. the first is A, the second B, etc.) and
%%a title (if you include one).  So, if you need
%hierarchical structure
%\textit{within} an Appendix, start with \textbf{subsection} as the
%highest level. Here is an outline of the body of this
%document in Appendix-appropriate form:
%\subsection{Introduction}
%\subsection{The Body of the Paper}
%\subsubsection{Type Changes and  Special Characters}
%\subsubsection{Math Equations}
%\paragraph{Inline (In-text) Equations}
%\paragraph{Display Equations}
%\subsubsection{Citations}
%\subsubsection{Tables}
%\subsubsection{Figures}
%\subsubsection{Theorem-like Constructs}
%\subsubsection*{A Caveat for the \TeX\ Expert}
%\subsection{Conclusions}
%\subsection{Acknowledgments}
%\subsection{Additional Authors}
%This section is inserted by \LaTeX; you do not insert it.
%You just add the names and information in the
%\texttt{{\char'134}additionalauthors} command at the start
%of the document.
%\subsection{References}
%Generated by bibtex from your ~.bib file.  Run latex,
%then bibtex, then latex twice (to resolve references)
%to create the ~.bbl file.  Insert that ~.bbl file into
%the .tex source file and comment out
%the command \texttt{{\char'134}thebibliography}.
% This next section command marks the start of
% Appendix B, and does not continue the present hierarchy
%\section{More Help for the Hardy}
%The acm\_proc\_article-sp document class file itself is chock-full of succinct
%and helpful comments.  If you consider yourself a moderately
%experienced to expert user of \LaTeX, you may find reading
%it useful but please remember not to change it.
\balancecolumns
% That's all folks!
\end{document}
