\newpage
\section{Introduction}
OpenStreetMap is a map built by its users. Users go out to track roads and
other objects with GPS positioning and then adds that information together with
interesting metadata to the OpenStreetMap project. This information is then
available for all its users under a free and open license. The process of
gathering GPS positions for all objects in the world is quite time consuming
so instead of tracking objects by foot some aerial imagery sources are
available so users can stay at home and add objects to the project with the
aerial imagery as source.

Even to add objects manually from aerial images can sometimes be cumbersome and
tedious. This process can be automated and let the user approve with or
without small modications the end result. This would free up time for the
users to spend on other mapping activities such as adding house numbers,
locations of emergency services and other points of interests.

Another purpose to motivate this work is the problem of alignment for
different image sets. The entire world can not fit in a single image
and still include enough detail to see roads, houses and other objects.
Instead stitching is applied where e.g. fixed-wing airplanes fly at a fixed
altitude and in a constant speed and takes several images over a large area.
These images are then "stitched" together to create a large map with enough detail.

In theory this would create a perfect map over a large area though in practice
several problems arise. Some aerial images are photographed at an angle and
creates misalignment cause of this. Other image sets comes with erroneous metadata
such as wrong GPS positions. With GPS traces of roads already added to OpenStreetMap
and automatic road recognition from the aerial imagery it would be possible to
set individual offset for all images to match the positioning of GPS traces.

A third motivating factor is somewhat related to the first problem mentioned above.
When high detail images over areas which have not been mapped before (or old
images are updated with increased detail) are added to sources like Bing maps
it can take sometime before anyone notices the new image. If an automated
process could look through all new images and send notifications to active
users it could increase response time.

% Another usage would be to create a network of roads as node and map that to
% already traced GPS nodes/networks. This would give the possibility to
% automatically align misaligned aerial images for the OpenStreetMap project.

\begin{itemize}
% OpenStreetMap (OSM) is a crowdsourced map built by users collecting gps-traces
%and tracing objects through a few aerial imagery sources that are available.
%...something about JOSM...
%...about this report...
%...The structure of this report...
% In Appendix more thourough look at OpenStreetMap and Java OpenStreetMap Editor
% is given.
% In Appendix "How to use implementation"
% In Appendix "Algorithm recap, with suggested values for application"
%
% Write about 4/8/Mixed choices and how that can affect.
% Define Tau for my use case and explain in the general case.
%
% ...notes while reading medical phd report...
% Goal with this project is a fully automatic process, no human interaction.
    \item Background OSM why use bing maps? What are they used for? if no or only
        a few gps traces exists they can be used to improve the detail of the 
        open street map.
    \item Problems with using Bing maps?
    \item Images are not photgraphed straight above, this gives us different problems.
        \begin{itemize}
            \item Curvy roads
            \item misalignment
            \item
        \end{itemize}
    \item Idea, how to use bing maps, three different questions.
        \begin{itemize}
            \item Automatically calculate bing aerial imagery offset.
            \item
            \item
        \end{itemize}
    \item So alot of problems with Bing maps and image analysis exists
        we are only to look at the image segmentation problem
        and why speed matters in this application.
    \item
    \item
    \item
\end{itemize}

