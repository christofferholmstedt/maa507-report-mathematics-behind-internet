\section{Conclusions}
A big problem within the field of image segmentation is to define and judge
what a good segmentation is. If an algorithm identifies one part of an image
in the foreground while another identfies something else in the background, which
one is better is hard to say. What can be said is when the result is not adequate
for post processing such as road recognition in this case.

The image over Lule\.{a} clearly shows some roads while other parts of the
image are more vague even for the human eye. The segmentation gives some hints
to what could be roads but the detail is not enough for future work to be worth it.
The second image which basically is a highway with a junction through a forest
should be relatively easy to segment. Segmentation of the highway image shows
the road going through the central parts of the image and can be defined as
a good segmentation if it's only the highway that is to be identified. The smaller
roads are near impossible to see in the segmentation.

The two images tested in this case shows that the Efficient Graph-Based Image Segmentation
is not viable for this application. There may be future work that can be done
to create a better threshold function to detect thinner lines.

This project started out with the algorithm and application, the result as stated
above is not satisfactory so future work should start with more specific algorithms
for road recognition. Example of work that can be intresting to look closer to
is listed below.
\begin{itemize}
  \item {\em "Automatic Road Extraction from Aerial Images"} by John C. Trinder and
Yandong Wang \cite{trinder1998}.
  \item {\em "Detection of Roads in Aerial Images by
Using Edge Information"} by Wu-Ja Lin and Chih-Wei Tseng \cite{lin2012}.
  \item {\em "Urban Road Extraction from High-Resolution Optical Satellite Images"} by
Mohamed Naouai, Atef Hamouda and Christiane Weber \cite{naouai2010}.
  \item {\em "Unsupervised line network extraction in remote sensing using a polyline
      process"} by Caroline Lacoste, Xavier Descombes and Josiane Zerubia \cite{lacoste2009}.
\end{itemize}

% The eye sees some different shapes that is not there we make them up in our
% mind.
% \nocite{*}
